\documentclass[a4paper,12pt]{article}


\usepackage[MeX]{polski}
\usepackage[utf8]{inputenc}	% kodowanie znaków
\usepackage{indentfirst}

\usepackage{upquote}  	% zamiana cudzysłowów klasycznych (,, ") na górne (" ") w kodach źródłowych
\usepackage{graphicx}	% wstawianie obrazków
\usepackage{amsmath}

% Ustawienie marginesów
\oddsidemargin=0.5cm
\evensidemargin=-0.5cm
\topmargin=0cm
\textwidth=16cm
\textheight=23cm

% Ładniejsze tabelki
\usepackage{booktabs}


\frenchspacing

\clubpenalty=10000		% to kara za sierotki
\widowpenalty=10000		% nie pozostawia wdów
\brokenpenalty=10000 	% nie dzieli wyrazów pomiędzy stronami


\sloppy

% Pakiety z ładnymi czcionkami
\usepackage[T1]{fontenc}
\usepackage{lmodern}


% Ładniejsze tabelki
\usepackage{booktabs}

\begin{document}

\title{{\small Programowanie równoległe i rozproszone}\\Algorytm szyfrujący AES w trybie CTR}
\author{Maciej Stefańczyk, Kacper Szkudlarek}

\maketitle

\begin{abstract}
Algorytm szyfrujący AES (Advanced Encryption Standard) roboczo nazywany
Rijndael. Jest to symetryczny algorytm z kluczem. Szyfruje i deszyfruje dane w 128-
bitowych blokach, korzystając z 128, 192 lub 256 bitowych kluczy. Należy zapoznać się
z algorytmem, zrealizować w wersji sekwencyjnej, w trybie szyfrowania CTR (Counter
mode). Zrównoleglenie zostało oparte o bibliotekę OpenMP dla pamięci wspólnej, oraz
MPI dla pamięci rozproszonej.
\end{abstract}


\section{Wstęp}

\section{Implementacja}

\subsection{Wersja sekwencyjna}

\subsection{Pamięć wspólna -- OpenMP}

\subsection{Pamięć rozproszona -- MPI}

\subsection{Dodatkowe możliwości}

\subsection{Obsługa programu}

\section{Testy}

\subsection{Testy jednostkowe}

\subsection{Testy wydajności}

\subsubsection{Dane testowe}
\subsubsection{Wyniki}







\begin{table}[h!]
\caption[Zadanie pierwsze -- czas przetwarzania pojedynczej ramki obrazu]{Czas przetwarzania pojedynczej ramki obrazu}
\centering
\begin{tabular}{lcccccccc}
\toprule
 & \multicolumn{2}{c}{1280x720} & \multicolumn{2}{c}{960x540} & \multicolumn{2}{c}{640x360} & \multicolumn{2}{c}{320x180} \\
\cmidrule(r){2-9}
 & czas [ms] & FPS & czas [ms] & FPS & czas [ms] & FPS & czas [ms] & FPS \\
\midrule
FraDIA & \bf 45,7 & 21,9 & \bf 26,8 & 37,3 & \bf 12,7 & 78,7 & \bf 4,5 & 222,2 \\
RAW    & \bf 44,5 & 22,5 & \bf 26,1 & 38,3 & \bf 12,1 & 82,6 & \bf 4,3 & 232,6 \\
\bottomrule
\end{tabular}
\label{tab:zad_1_wyniki}
\end{table}

\end{document}
